\chapter{domains-Module}

Domains specify the domains on which the \modst{morsaik}-objects live.
They ensure compatibility and consistence of implemented code.
This is especially important for the inference part
when performed with the \modst{NIFTy} package \cite{Edenhofer2024NIFTyre}.
Typically, shapes and units of \modst{morsaik}-objects are specified here.
We define the following domains:

\begin{itemize}
    \item Index set $\Idx$\\
        In general, not implemented explicitly in the
        \modst{morsaik}-package,
        but implicitly captured by shapes of objects.
    \item Time Space $\timeSpace \in \{ \R, \R^n \}$\\
        Typically used for specifying the time points of measurements.
    \item Number Space $\numberSpace_u \cong \No$ for all units $u$\\
        If for example motifs or reactions are counted explicitly
    \item Concentration Space $\concentrationSpace_u \cong \nnR$ for all units
        $u$\\
        For example for concentrations of motifs or average reactions
    \item Hamming Space $\hammingSpace$\\
        Set of all motifs with letters from the alphabet
        $\alphabet$ and length $\motiflength$
    \item Motif Space
        $\motifSpace = \bigoplus_{\motiflength' \leq \motiflength} \hammingspace_{\alphabet,\motiflength'}$\\
        Set of all strands or motifs with letters from the alphabet
        $\alphabet$ and maximum length $\motiflength$
    \item Motif Hyphen Space $\motifHyphenSpace 
        = \bigoplus_{\motiflength' \in \{2,\dots,\motiflength\}}
        \bigoplus_{i \in \{1, \dots, \motiflength'-1\} }
        \hammingspace_{\alphabet,\motiflength'}
        = \bigoplus_{\motiflength' \in \{2,\dots,\motiflength\}}
        \hammingspace_{\alphabet,\motiflength'}^{\motiflength'-1}$\\
        Set of motif specific covalent bonds with index $i$ indicating the
        position of the covalent bond in the $\motiflength'$-mer.
    \item Extended Motif Production Space $\widetilde \motifProductionSpace
        = \motifSpace
        \otimes \motifSpace
        \otimes \motifHyphenSpace$\\
        The first motif space specifies the (left) ending reactant, the second
        motif space the (right) beginning reactant, the third space is the
        motif coupling vector space that indicates the template and its
        covalent bound.
        Here, $\Idx$ sets the index of the ligation spot on the template,
        which is a number between one and the length of the template minus one.
        \\
        The reduced motif production space $\motifProductionSpace
        = \motifHyphenSpace
        \otimes \motifHyphenSpace$
        is specified by the central product and the template motif
        that overlaps the most with the central product.
        In this reduced space, one marginalizes out the dangling ends of the
        reactants.
    \item Extended Motif Breakage Space $\widetilde \motifBreakageSpace = \motifSpace \otimes
        \motifSpace$ is specified by the motif space of the left broken and the
        right broken motif.
        \\
        The reduced breakage space is specified by the breaking motif and thus
        is less dimensional:
        $\motifBreakageSpace =\motifHyphenSpace$
        here, $\Idx$ sets the index of the breakage spot
        thus, a number between one and the length of the breaking motif minus one.
\end{itemize}
Typically, units can be assigned to the domains, for example
\begin{itemize}
    \item time: $s$
    \item concentrations: $\mathrm{mol}/\mathrm L$
    \item numbers: $1$ or mol specifically
    \item motif production rate constant: $\left( \mathrm L / \mathrm{mol} \right)^2$
    \item breakage rate constants: $\mathrm{mol}/\mathrm{L}$
\end{itemize}
