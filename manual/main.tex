\chapter{Main Module}

The \modst{morsaik}-code has been developed using test-driven development.
A huge base of tests is given in  the \modst{test}-directory of the \modst{morsaik} repository.
For an overview, please refer to the paper in the paper directory of the git
repository.
In addition, an API documentation exists in the docs directory in the git
repository.
This manual shell provide deeper insides into the design choices and,
how different modules are combined together.

The package contains five submodules: \modst{domains}, \modst{read},
\modst{infer}, \modst{get} and \modst{utils}, see Figure~\ref{fig:submodules},
which will be discussed in
separate chapters in the following.
Additionally, the objects of the package are discussed directly after the
\modst{domains} submodule, as they are typical inputs of the package functions.

\begin{figure}
    \centering
    \begin{tikzpicture}[
            modulenode0/.style={double,rounded corners, draw=black!60, fill=white!5, very thick, minimum size=3mm},
            modulenode1/.style={double,rounded corners, draw=black!60, fill=white!5, thick, minimum size=3mm},
            objnode1/.style={ellipse, draw=black!60, fill=white!5, very thick, minimum size=3mm},
            opnode1/.style={rectangle, draw=black!60, fill=white!5, very thick, minimum size=3mm},
        ]
        %Nodes
        \node[modulenode0]  (morsaiknode) {morsaik};
        \node[modulenode1]  (objnode)  [below =of morsaiknode] {obj};
        \node[modulenode1]  (domainsnode)  [below left =of morsaiknode] {domains};
        \node[modulenode1]  (readnode)  [above left=of morsaiknode] {read};
        \node[modulenode1]  (infernode)  [above =of morsaiknode] {infer};
        \node[modulenode1]  (utilsnode)  [below right=of morsaiknode] {utils};
        \node[modulenode1]  (getnode)  [above right=of morsaiknode] {get};
        %Lines
        \draw[-, very thick] (morsaiknode.south) -- (objnode.north);
        \draw[-, very thick] (morsaiknode.south west) -- (domainsnode.north east);
        \draw[-, very thick] (morsaiknode.north west) -- (readnode.south east);
        \draw[-, very thick] (morsaiknode.north) -- (infernode.south);
        \draw[-, very thick] (morsaiknode.north east) -- (getnode.south west);
        \draw[-, very thick] (morsaiknode.south east) -- (utilsnode.north west);
    \end{tikzpicture}
    \caption{Graphical representation of the inference module and iwhichts submodules.}
    \label{fig:submodules}
\end{figure}


\chapter{domains-Module}

Domains specify the domains on which the \modst{morsaik}-objects live.
They ensure compatibility and consistence of implemented code.
This is especially important for the inference part
when performed with the \modst{NIFTy} package \cite{Edenhofer2024NIFTyre}.
Typically, shapes and units of \modst{morsaik}-objects are specified here.
We define the following domains:

\begin{itemize}
    \item Index set $\Idx$\\
        In general, not implemented explicitly in the
        \modst{morsaik}-package,
        but implicitly captured by shapes of objects.
    \item Time Space $\timeSpace \in \{ \R, \R^n \}$\\
        Typically used for specifying the time points of measurements.
    \item Number Space $\numberSpace_u \cong \No$ for all units $u$\\
        If for example motifs or reactions are counted explicitly
    \item Concentration Space $\concentrationSpace_u \cong \nnR$ for all units
        $u$\\
        For example for concentrations of motifs or average reactions
    \item Hamming Space $\hammingSpace$\\
        Set of all motifs with letters from the alphabet
        $\alphabet$ and length $\motiflength$
    \item Motif Space
        $\motifSpace = \bigoplus_{\motiflength' \leq \motiflength} \hammingspace_{\alphabet,\motiflength'}$\\
        Set of all strands or motifs with letters from the alphabet
        $\alphabet$ and maximum length $\motiflength$
    \item Motif Hyphen Space $\motifHyphenSpace 
        = \bigoplus_{\motiflength' \in \{2,\dots,\motiflength\}}
        \bigoplus_{i \in \{1, \dots, \motiflength'-1\} }
        \hammingspace_{\alphabet,\motiflength'}
        = \bigoplus_{\motiflength' \in \{2,\dots,\motiflength\}}
        \hammingspace_{\alphabet,\motiflength'}^{\motiflength'-1}$\\
        Set of motif specific covalent bonds with index $i$ indicating the
        position of the covalent bond in the $\motiflength'$-mer.
    \item Extended Motif Production Space $\widetilde \motifProductionSpace
        = \motifSpace
        \otimes \motifSpace
        \otimes \motifHyphenSpace$\\
        The first motif space specifies the (left) ending reactant, the second
        motif space the (right) beginning reactant, the third space is the
        motif coupling vector space that indicates the template and its
        covalent bound.
        Here, $\Idx$ sets the index of the ligation spot on the template,
        which is a number between one and the length of the template minus one.
        \\
        The reduced motif production space $\motifProductionSpace
        = \motifHyphenSpace
        \otimes \motifHyphenSpace$
        is specified by the central product and the template motif
        that overlaps the most with the central product.
        In this reduced space, one marginalizes out the dangling ends of the
        reactants.
    \item Extended Motif Breakage Space $\widetilde \motifBreakageSpace = \motifSpace \otimes
        \motifSpace$ is specified by the motif space of the left broken and the
        right broken motif.
        \\
        The reduced breakage space is specified by the breaking motif and thus
        is less dimensional:
        $\motifBreakageSpace =\motifHyphenSpace$
        here, $\Idx$ sets the index of the breakage spot
        thus, a number between one and the length of the breaking motif minus one.
\end{itemize}
Typically, units can be assigned to the domains, for example
\begin{itemize}
    \item time: $s$
    \item concentrations: $\mathrm{mol}/\mathrm L$
    \item numbers: $1$ or mol specifically
    \item motif production rate constant: $\left( \mathrm L / \mathrm{mol} \right)^2$
    \item breakage rate constants: $\mathrm{mol}/\mathrm{L}$
\end{itemize}

\chapter{morsaik-Objects}

Given the domains in the last section,
we can now define typical \modst{morsaik}-objects.
\begin{enumerate}
    \item motif vector: $\motifSpace \rightarrow O \in \{\numberSpace_{u}, \concentrationSpace_u\}$ with some unit $u$.\\
        Vector of motif concentrations or numbers,
        see Section~\ref{sec:MotifVector}
        (unit specified by keyword unit):\\
        motif number vector: $\motifSpace \rightarrow \numberSpace_{[1]}$, and\\
        motif concentration vector: $\motifSpace \rightarrow \concentrationSpace_{[\mathrm{mol}/\mathrm{L}]}$
    \item motif trajectory: $\motifSpace \otimes \timeSpace \rightarrow O \in \{\numberSpace_{u}, \concentrationSpace_u\}$ with some unit $u$.\\
        Time series of motif vectors, expecially\\
        motif number trajectory: $\motifSpace \otimes \timeSpace \rightarrow \numberSpace_{[1]}$, and\\
        motif concentration trajectory: $\motifSpace \otimes \timeSpace \rightarrow \concentrationSpace_{[\mathrm{mol}/\mathrm{L}]}$
    \item motif trajectory ensemble: $\motifSpace \otimes \timeSpace \otimes \Idx \rightarrow O \in \{\numberSpace_{u}, \concentrationSpace_u\}$ with some unit $u$.\\
        Ensemble of motif trajectories,//
        motif number trajectory ensemble: $\motifSpace \otimes \timeSpace \otimes \Idx \rightarrow \numberSpace_{[1]}$, and\\
        motif concentration trajectory ensemble: $\motifSpace \otimes \timeSpace \otimes \Idx \rightarrow \concentrationSpace_{[\mathrm{mol}/\mathrm{L}]}$
    \item motif production vector: $\motifProductionSpace \rightarrow \numberSpace_{[1]}$\\
        Vector of motif productions,
        specified by the product motif,
        the template motif,
        the ligation window and
        the location of the ligation inside this ligation window,
        see Section\ref{sec:MotifProductionVector}
    \item motif production trajectory: $\motifProductionSpace \otimes \timeSpace \rightarrow \numberSpace_{[1]}$\\
        Time series of motif production vectors.
    \item motif production trajectory ensemble: $\motifProductionSpace \otimes \timeSpace \otimes \Idx \rightarrow \numberSpace_{[1]}$\\
        Ensemble of motif production trajectories.
    \item motif production rate constant: $\motifProductionSpace \rightarrow \concentrationSpace_{[\mathrm{L}^2/\mathrm{mol}^2]}$
    \item motif breakage vector: $\motifBreakageSpace \rightarrow \numberSpace_{[1]}$\\
        Vector of motif breakages,
        specified by the left and the right broken motif.
    \item motif breakage trajectory: $\motifBreakageSpace \otimes \timeSpace \rightarrow \numberSpace_{[1]}$\\
        Time series of of motif breakage vectors.
    \item motif breakage trajectory ensemble: $\motifBreakageSpace \otimes \timeSpace \otimes \Idx \rightarrow \numberSpace_{[1]}$\\
        Ensemble of motif breakage trajectories.
    \item motif breakage rate constant: $\motifBreakageSpace \rightarrow \concentrationSpace_{[\mathrm{mol}/\mathrm{L}]}$\\
\end{enumerate}

\section{Motif Vector}
\label{sec:MotifVector}

As a parent class for motif number vector and motif concentration vector serves
the abstract class motif vector.
Its domain is the same like for motif number and motif concentration vector,
however its unit is unspecified.
For convenience, function to save and load \modst{MotifVector}-objects on
respectively from the disk are implemented, as well as functions that compare
\modst{MotifVector}-domains, for example to build trajectories, see
Figure~\ref{fig:MotifVector}.
For computations, \modst{MotifVectors} can be transformed to \modst{jax.Array}s
(and back), see Figure~\ref{fig:MotifVectorTransformations}.

\begin{figure}
    \centering
    \begin{tikzpicture}[
            modulenode1/.style={double,rounded corners, draw=black!60, fill=white!5, thick, minimum size=3mm},
            objnode1/.style={ellipse, draw=black!60, fill=white!5, very thick, minimum size=3mm},
            opnode1/.style={rectangle, draw=black!60, fill=white!5, very thick, minimum size=3mm},
        ]
        %Nodes
        \node[objnode1]  (MVnode) {MotifVector};
        \node[opnode1]  (savenode)  [right =of MVnode] {$\mathsf{save\_motif\_vector}$};
        \node[opnode1]  (loadnode)  [left =of MVnode] {$\mathsf{load\_motif\_vector}$};
        \node[opnode1]  (instancenode)  [below=of loadnode] {$\mathsf{isinstance\_motifvector}$};
        \node[opnode1]  (compatiblenode)  [below=of savenode] {$\mathsf{are\_compatible\_motif\_vectors}$};
        \node[objnode1] (bool1node) [below=of instancenode] {$\mathsf{boolean}$};
        \node[objnode1] (bool2node) [below=of compatiblenode] {$\mathsf{boolean}$};
        \node[objnode1] (pathnode) [above=of MVnode] {$\mathsf{file\_path}$:$\mathsf{string}$};
        \node[objnode1] (filenode) [above=of savenode] {$\mathsf{file}$};
        %Lines
        \draw[->, very thick] (loadnode.east) -- (MVnode.west);
        \draw[->, very thick] (pathnode.south west) -- (loadnode.north east);
        \draw[->, very thick] (MVnode.east) -- (savenode.west);
        \draw[->, very thick] (pathnode.south east) -- (savenode.north west);
        \draw[->, very thick] (savenode.north) -- (filenode.south);
        \draw[->, very thick] (MVnode.south west) -- (instancenode.north east);
        \draw[->, very thick] (MVnode.south east) -- (compatiblenode.north);
        \draw[->, very thick] (MVnode.south) -- (compatiblenode.north west);
        \draw[->, very thick] (compatiblenode.south) -- (bool2node.north);
        \draw[->, very thick] (instancenode.south) -- (bool1node.north);
    \end{tikzpicture}
    \caption{Motif Vector object and related functions.}
    \label{fig:MotifVector}
\end{figure}
\begin{figure}
    \centering
    \begin{tikzpicture}[
            modulenode1/.style={double,rounded corners, draw=black!60, fill=white!5, thick, minimum size=3mm},
            objnode1/.style={ellipse, draw=black!60, fill=white!5, very thick, minimum size=3mm},
            opnode1/.style={rectangle, draw=black!60, fill=white!5, very thick, minimum size=3mm},
            nonode1/.style={},
        ]
        \node[nonode1] (centernode) {};
        \node[objnode1]  (MVnode) [left=of centernode] {MotifVector};
        \node[opnode1]  (vector2arraynode)  [above =of centernode] {$\mathsf{\_motif\_vector\_as\_array}$};
        \node[opnode1]  (array2vectornode)  [below =of centernode] {$\mathsf{\_array\_to\_motif\_vector\_dct}$};
        \node[objnode1]  (manode) [right =of centernode] {$\mathsf{motif\_array:numpy.ndarray}$};
        \node[opnode1]  (sequence2arraynode)  [above =of vector2arraynode] {$\mathsf{\_transform\_sequence\_array\_to\_motif\_array}$};
        \node[opnode1]  (createnode)  [left=of MVnode] {$\mathsf{\_create\_empty\_motif\_vector\_dct}$};
        \node[opnode1]  (indicesnode)  [above =of manode] {$\mathsf{\_motif\_indices\_in\_motifs\_array}$};
        %Lines
        \draw[->, very thick] (MVnode.north) -- (vector2arraynode.south west);
        \draw[->, very thick] (createnode.east) -- (MVnode.west);
        \draw[->] (vector2arraynode.north) -- (sequence2arraynode.south west);
        \draw[->] (sequence2arraynode.south east) -- (vector2arraynode.north);
        \draw[->, very thick] (vector2arraynode.south east) -- (manode.north west);
        \draw[->, very thick] (manode.south west) -- (array2vectornode.north east);
        \draw[->, very thick] (array2vectornode.north west) -- (MVnode.south);
    \end{tikzpicture}
    \caption{The transformation of a Motif Vector to a numpy array and back.}
    \label{fig:MotifVectorTransformations}
\end{figure}

\section{Motif Production Vector}
\label{sec:MotifProductionVector}

The motif production vector carries the motif productions in form of a named
tuple with properties whose keys are built as
$\mathsf{<product\_key>}\_\mathsf{<ligation\_window\_length>}\_\mathsf{<ligation\_spot>}\_\mathsf{<template\_key>}$.
Each of these reactions are specified by an array of the size of the number of
nucleotides involved in the reaction.
Ends are not captured as they are already fully specified by the
$\mathsf{<product\_key>}$ and the $\mathsf{<template\_key>}$,
their relative position is specified by the 
$\mathsf{<ligation\_window\_length>}$ and the $\mathsf{<ligation\_spot>}$:
The $\mathsf{<ligation\_window\_length>}$ is chosen such that it captures all
hybridized nucleotides and at maximum one more dangling or vertical end at each
side.

The $\mathsf{<ligation\_spot>}$ is a natural number between one and 
$\left(\mathsf{<ligation\_window\_length>}-2\right)$, where one encodes the ligation after
the first two nucleotide spots in the ligation window and
$\left(\mathsf{<ligation\_window\_length>}-2\right)$ the ligation two nucleotides before the
end of the ligation window end.

The $\mathsf{<ligation\_spot>}$ being zero is theoretically possible, this
would mean that the ligation happens directly after the first nucleotide spot.
By definition, there is no ligation window that captures such a reaction 
-- only if the $\mathsf{<maximum\_ligation\_window\_length>}$ is set to two, then
$\mathsf{<ligation\_spot>}$ is always zero.

For all $\mathsf{<maximum\_ligation\_window\_length>}>2$, the zeroth ligation
spot is either a dangling end, which does not allow templated ligation, or a
continuation to the corresponding side, however, then this reaction is captured
by another ligation window that is just moved further to the left.

An alternative representation of the productions is the representation as
$\mathsf{jax.Array}$, which is just called motif array in the following.
Its format is arranged such that the ligation spot is always in the
center of the $\mathsf{<maximum\_ligation\_window\_length>}$.
Periodic boundary conditions enable to track the whole ligation windows:

\paragraph{Example:}
Let the monomer `$\mathtt{a}$' ligate to the tetramer `$\mathtt{tta}$' on the
(beginning) template `$\mathtt{tatat}$'.
This reaction is fully captured by a ligation window of length 6, with a
dangling end at the left hand side and a vertical end at the right hand side.
We state the ligation spot on the template with a ``$\text-$'',
the ligation between the two reactants with ``$|$''
and seperate the product from the template with a ``$:$''.
The described reaction would then be noted as
`$\mathtt{a|tta:tat\text{-}at}$'.

In the array format, we explicitly note ends (or beginnings) with zeros.
Then the reaction is denoted as `$\mathtt{0a|tta0:0tat\text-at}$'.
For compactness, we choose the ligation in the array always to happen in the
center of the ligation window.
Additionally, we imply periodic boundary conditions, 
to track the continuation of the product outside the ligation window and save
unnecessary zeros. 
The end of the ligation window is then specified by a zero either in the
template or in the product motif or in both.
With such periodic boundary conditions and the ligation spot in the center, the
array-format becomes
`$\mathtt{0;0a|tta:tat\text-at;0}$'.
The array for a ligation window of length six is defined on the domain
$\alphabet_0^2 \otimes \alphabet^2 \otimes \alphabet_0 \bigotimes \alphabet_0
\otimes \alphabet^2 \otimes \alphabet_0^2$.
The indices at the ligation spot always indicate letters (never empty spots),
thus their index 0 denotes the first letter, where at all other ligation spots
0 actually denotes an empty spot.
With this, we finally find the array format of the same reaction as
$(0;0,0|1,2,1:2,1,1\text-0,2;0)$.
The motif production array in $\mathsf{jax.Array}$-format saves this reaction
with the index tuple $(0,0,0,1,2,1,2,1,1,0,2,0)$.

\chapter{read-Module}

The \modst{read}-module is responsible for reading files from disk,
especially rna strand reactor output files.
For an example, see demo \modst{1\_read\_strand\_reactor\_data.py}.

\chapter{infer-Module}

The infer module is responsible for inferring variable from data.
Different models are implemented here that allow different inferences.
Typical inference functions use Bayesian inference (using \modst{NIFTy.re}),
but there are also infer-functions that naively compute one variable from the
other in a deterministic, non-statistical fashion,
as well as functions that simulate dynamics,
such as motif dynamics,
given the parameters of those dynamics.
For an overview of main concept in the \modst{infer} module see
Figure~\ref{fig:InferSubmod},
for an example, Figure~\ref{fig:MotifTrajectoryFromRateConstants}.
Note that in the figures, names of functions might be abbreviated or more
abstract.
For the exact name of the function, please refer to the documentation.
\begin{figure}
    \centering
    \rotatebox{90}{
    \begin{tikzpicture}[
            modulenode1/.style={double,rounded corners, draw=black!60, fill=white!5, thick, minimum size=3mm},
            objnode1/.style={ellipse, draw=black!60, fill=white!5, very thick, minimum size=3mm},
            opnode1/.style={rectangle, draw=black!60, fill=white!5, very thick, minimum size=3mm},
        ]
        %Nodes
        \node[modulenode1]  (infernode)  {infer};
        \node[opnode1] (rates2motifsnode) [right=of infernode] {$\mathsf{motif\_concentration\_trajectory\_from\_rate\_constants}$};
        \node[opnode1] (concentrations2rateconstantsnode) [above left=of infernode] {$\mathsf{rate\_constants\_from\_motif\_concentration\_trajectories}$};
        \node[opnode1] (counts2rateconstantsnode) [left=of infernode] {$\mathsf{rate\_constants\_from\_motif\_production\_trajectories}$};
        \node[opnode1] (parameters2rateconstantsnode) [below left=of infernode] {$\mathsf{rate\_constants\_from\_strand\_reactor\_parameters}$};
        \node[objnode1] (rcnode) [below=of rates2motifsnode] {$\mathsf{rate\_constants}$};
        \node[objnode1] (mtnode) [above=of rates2motifsnode] {$\mathsf{MotifTrajectoryEnsemble}$};
        %Lines
        \draw[-, very thick] (infernode.east) -- (rates2motifsnode.west);
        \draw[-, very thick] (infernode.west) -- (counts2rateconstantsnode.east);
        \draw[-, very thick] (infernode.north west) -- (concentrations2rateconstantsnode.south east);
        \draw[->, very thick] (counts2rateconstantsnode.south east) -- (rcnode.west);
        \draw[->, very thick] (concentrations2rateconstantsnode.east) -- (rcnode.north west);
        \draw[->, very thick] (parameters2rateconstantsnode.east) -- (rcnode.west);
        \draw[-, very thick] (infernode.south west) -- (parameters2rateconstantsnode.north east);
        \draw[->, very thick] (rcnode.north) -- (rates2motifsnode.south);
        \draw[->, very thick] (rates2motifsnode.north) -- (mtnode.south);
        \draw[->, very thick] (mtnode.west) -- (concentrations2rateconstantsnode.north east);
    \end{tikzpicture}
    }
    \caption{Graphical representation of the inference submodule and its main concepts.}
    \label{fig:InferSubmod}
\end{figure}

%\subsubsection{Infer Motif Trajectories from Rate Constants}

\begin{figure}
    \centering
    \rotatebox{90}{
    \begin{tikzpicture}[
            modulenode1/.style={double,rounded corners, draw=black!60, fill=white!5, thick, minimum size=3mm},
            objnode1/.style={ellipse, draw=black!60, fill=white!5, very thick, minimum size=3mm},
            objnode2/.style={ellipse, draw=black!60, fill=white!5, thick, minimum size=3mm},
            opnode1/.style={rectangle, draw=black!60, fill=white!5, very thick, minimum size=3mm},
            opnode2/.style={rectangle, draw=black!60, fill=white!5, thick, minimum size=3mm},
        ]
        %Nodes
        \node[opnode1] (rates2motifsnode) {$\mathsf{infer.fourmer\_trajectory\_from\_rate\_constants}$};
        \node[objnode2] (timesnode) [above left=of rates2motifsnode]{$\mathsf{times : TimesVector}$};
        \node[objnode2] (inimocove) [above =of timesnode]{$\mathsf{initial\_motif\_concentrations\_vector : MotifVector}$};
        \node[objnode2] (brearaconode) [above =of inimocove]{$\mathsf{breakage\_rate\_constants : MotifBreakageVector}$};
        \node[objnode2] (moprorateconode) [above=of brearaconode]{$\mathsf{motif\_production\_rate\_constants : MotifProductionVector}$};
        \node[objnode2] (complnode) [left=of rates2motifsnode]{$\mathsf{complements : list}$};
        \node[objnode2] (odeintmethnode) [below left=of
        rates2motifsnode]{$\mathsf{ode\_integration\_method:str=``BDF"}$};
        \node[objnode2] (exectimenode) [below =of odeintmethnode]{$\mathsf{execution\_time\_path:str=None}$};
        \node[objnode2] (motra) [right=of rates2motifsnode]{$\mathsf{MotifConcentrationTrajectory}$};
        %Lines
        \draw[->, thick] (moprorateconode.east) -- (rates2motifsnode.north);
        \draw[->, thick] (brearaconode.east) -- (rates2motifsnode.north);
        \draw[->, thick] (inimocove.east) -- (rates2motifsnode.north);
        \draw[->, thick] (timesnode.east) -- (rates2motifsnode.north west);
        \draw[->, thick] (complnode.east) -- (rates2motifsnode.west);
        \draw[->, thick] (odeintmethnode.east) -- (rates2motifsnode.south);
        \draw[->, thick] (exectimenode.east) -- (rates2motifsnode.south);
        \draw[->, thick] (rates2motifsnode.east) -- (motra.west);
    \end{tikzpicture}
    }
    \caption{Graphical representation of the $\mathsf{infer.motif\_trajectory\_from\_rate\_constants}$ function.}
    \label{fig:MotifTrajectoryFromRateConstants}
\end{figure}

\chapter{get-Module}

The \modst{get} submodule is a convenience submodule combining the \modst{read}- and the
\modst{infer}-submodules.
Giving the id of the object, one wants to get, it looks, whether it can read
an already computed version from disk and give you the corresponding object.
If there is no saved version, it infers (sometimes simulates) the object asked
for from the information it has.
For each id, a yaml file specifies this information.
The yaml files are saved in the \modst{config}-directory for parameters
specified by the user or in the \modst{archive}-directory for saving computed
arrays.

\chapter{utils-Module}

In the \modst{utils}-submodule you find all utility functions responsible for managing
files, splitting config files, etc..


\newpage
\printnoidxglossary[type=symbols,style=long,title={List of Symbols}]
